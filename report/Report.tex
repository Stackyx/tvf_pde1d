\documentclass[11pt]{article}
\usepackage{fullpage}
\usepackage[english]{babel} %frenchb]{babel}
\usepackage[usenames]{color} %pour la couleur
\usepackage{amssymb} %maths
\usepackage{amsmath} %maths
\usepackage{mathabx}
\usepackage[latin1]{inputenc} %utile pour taper directement les caractères accentués
\usepackage[T1]{fontenc} 
\usepackage{hyperref}
\newtheorem{theorem}{Th�or�me}[section]
\newtheorem{lemma}{Lemme}[section]
\newtheorem{corollary}{Corollaire}[section]
\newtheorem{example}[theorem]{Exemple}
\newtheorem{definition}[theorem]{D�finition}
\usepackage[]{amsmath, amssymb, tabularx}
\usepackage{eurosym}
%\include{MacrosJEP}
\usepackage{float}

\input xy
\xyoption{all}

\usepackage{tikz} 
\usetikzlibrary{automata,calc}

\usepackage{xspace}

\setcounter{MaxMatrixCols}{15}

\title{C++ Project \\ MSc 203 \\ University Paris Dauphine}

\author{Thibaut JEREBITZ \\ Florian PERIGNE \\ Valentin ROCHEREAU}

\begin{document}
%%%%%%%%%%%%%%%%%%%%%%%%%%%%%%%%%%%%%%%%%%%%%%%%%%%%%


%%%%%%%%%%%%%%%%%%%%%%%%%%%%%%%%%%%%%%%%%%%%%%%%%%%%%
\begin{titlepage}

\begin{figure}[H]
\begin{center}
\includegraphics[height=3cm]{Logo_Dauphine.png}
\end{center}
\end{figure}

\vspace*{\fill}

\centering \Huge
MSc 203 \\ 
University Paris Dauphine\\
\vspace*{\fill}
\normalsize

\begin{center}
\LARGE
\textbf{C++ Project }\\  
\end{center}
\normalsize

\vspace*{\fill}

\textbf{Thibaut JEREBITZ \\ Florian PERIGNE \\ Valentin ROCHEREAU}




\end{titlepage}
%%%%%%%%%%%%%%%%%%%%%%%%%%%%%%%%%%%%%%%%%%%%%%%%%%%%%%%
%%%%%%%%%%%%%%%%%%%%%%%%%%%%%%%%%%%%%%%%%%%%%%%%%%%%%%%
\maketitle

\vspace*{\fill}

\begin{abstract} 

\begin{center}
The goal of the project is to solve the Black Scholes PDE. \\
In this document, we develop the mathematics needed to implement a C++ code to solve the PDE.
\end{center}

\end{abstract}
\vspace*{\fill}

\addcontentsline{toc}{section}{Abstract}
%%%%%%%%%%%%%%%%%%%%%%%%%%%%%%%%%%%%%%%%%%%%%%%%%%%%%%%
\newpage
\maketitle

\renewcommand{\contentsname}{Outlines} 
\tableofcontents


%%%%%%%%%%%%%%%%%%%%%%%%%%%%%%%%%%%%%%%%%%%%%%%%%%%%%%%
%%%%%%%%%%%%%%%%%%%%%%%%%%%%%%%%%%%%%%%%%%%%%%%%%%%%%%%
\newpage
\section{Introduction} 

The price of a European option satitisfies the \textit{Black Scholes equation}
\begin{gather*}
0 = \frac{\partial f}{\partial t}(s, t) + \frac{1}{2}\sigma^2 s^2 \frac{\partial^2 f}{\partial s^2}(s, t)
+ r s \frac{\partial f}{\partial s}(s, t) - rf(s, t)
\end{gather*}
with the final condition $f(s, T) = V(s)$ where $V(s)$ is the payout.\\

Applying the variable change $x = log(s)$ yields :
\begin{gather*}
\frac{\partial f}{\partial t}(x, t) = -\frac{1}{2}\sigma^2 \frac{\partial^2 f}{\partial x^2}(x, t)
+ (\frac{1}{2}\sigma^2 - r) \frac{\partial f}{\partial x}(x, t) + rf(x, t)
\end{gather*}

with the final condition $f(x, T) = V(x) = V(e^s)$.\\

A common method to solve this PDE is to use finite differences to approximate the derivatives and then
solve the equation on a discrete mesh.\\

When developping the different formula, we will have to set boundaries conditions for the points on the border of the mesh. We will develop two possible solutions to set these conditions : 
\begin{itemize}
	\item Dirichlet method (part \ref{Dpart} )
	\item Neumann method (part \ref{Npart} )
\end{itemize}


%%%%%%%%%%%%%%%%%%%%%%%%%%%%%%%%%%%%%%%%%%%%%%%%%%%%%%%
%%%%%%%%%%%%%%%%%%%%%%%%%%%%%%%%%%%%%%%%%%%%%%%%%%%%%%%
\newpage
\section{Dirichlet }\label{Dpart}

\indent $ \forall  n \in  \ldbrack  1 , N  \rdbrack , \forall  i \in  \ldbrack  1,x_{max}  \rdbrack, $\\

\begin{align}
f_{i}^{n} &= f(t_{n},x_{i}) \\
\frac{f_{i}^{n+1} - f_{i}^{n}}{dt} &= \theta L_{i}^{n}  + (1-\theta) L_{i}^{n+1} 
\end{align}

%\begin{align}
%\begin{split}
%f_{i}^{n+1} \left( \frac{1}{dt} - \frac{\sigma^2}{dx^2} (1- \theta) - r (1 - \theta) \right)  &+
%f_{i+1}^{n+1} \left( \frac{\sigma^2}{2dx^2} (1- \theta) -  (\frac{\sigma^2}{2}- r) (1- \theta) \frac{1}{2dx} \right)  \\
%+ f_{i-1}^{n+1} \left( \frac{\sigma^2}{2dx^2} (1- \theta) +  (\frac{\sigma^2}{2}- r) (1- \theta) \frac{1}{2dx} \right)  &=
%f_{i}^{n} \left( \frac{1}{dt} + \frac{\theta \sigma^2}{dx^2} + r \theta  \right) \\
%+ f_{i+1}^{n} \left( -\theta \frac{\sigma^2}{2dx^2} +  (\frac{\sigma^2}{2}- r) \theta \frac{1}{2dx} \right) 
%&+ f_{i-1}^{n} \left( -\theta \frac{\sigma^2}{2dx^2} - (\frac{\sigma^2}{2}- r) \theta \frac{1}{2dx} \right)  
%\end{split}\\
% f_{i}^{n+1} A_{1}^{(1-\theta)} +f_{i+1}^{n+1} B_{-}^{(1-\theta)}  +f_{i-1}^{n+1} B_{+}^{(1-\theta)} &=
% f_{i}^{n} A_{2}^{\theta} - f_{i+1}^{n} B_{-}^{\theta}  -f_{i-1}^{n} B_{+}^{\theta} 
% \end{align}
 
Let  $ A_{1}^{(1-\theta)}, B_{-}^{(1-\theta)}, B_{+}^{(1-\theta)}, A_{2}^{\theta}, B_{-}^{\theta} $ and $ B_{+}^{\theta} $ take the following values :
 
\begin{align}
\begin{split}
A_{1}^{(1-\theta)} &= \frac{1}{dt} - \frac{\sigma^2}{dx^2} (1- \theta) - r (1 - \theta)   \\
B_{-}^{(1-\theta)} &= \frac{\sigma^2}{2dx^2} (1- \theta) -  (\frac{\sigma^2}{2}- r) (1- \theta) \frac{1}{2dx}  \\
B_{+}^{(1-\theta)} &= \frac{\sigma^2}{2dx^2} (1- \theta) +  (\frac{\sigma^2}{2}- r) (1- \theta) \frac{1}{2dx} \\
A_{2}^{\theta} &=  \frac{1}{dt} + \frac{\theta \sigma^2}{dx^2} + r \theta  \\
B_{-}^{\theta} &= -\theta \frac{\sigma^2}{2dx^2} +  (\frac{\sigma^2}{2}- r) \theta \frac{1}{2dx} \\ 
B_{+}^{\theta} &= -\theta \frac{\sigma^2}{2dx^2} - (\frac{\sigma^2}{2}- r) \theta \frac{1}{2dx}   
\end{split}
\end{align}

Then we obtain :

\begin{align} 
 f_{i}^{n+1} A_{1}^{(1-\theta)} +f_{i+1}^{n+1} B_{-}^{(1-\theta)}  +f_{i-1}^{n+1} B_{+}^{(1-\theta)} &=
 f_{i}^{n} A_{2}^{\theta} - f_{i+1}^{n} B_{-}^{\theta}  -f_{i-1}^{n} B_{+}^{\theta} 
 \end{align}
 
 The previous formula gives us the following matrix system :
 
\begin{align}
\begin{split}
 \begin{pmatrix}
1 & 0 & 0 & 0 & 0 & \cdots & 0 & 0\\
B_{+}^{(1-\theta)} & A_{1}^{(1-\theta)} & B_{-}^{(1-\theta)} & 0 & 0 & \cdots & 0 & 0\\
0 & B_{+}^{(1-\theta)} & A_{1}^{(1-\theta)} & B_{-}^{(1-\theta)} & 0 & \cdots & 0 & 0 \\
\vdots& \vdots & \vdots & \vdots & \vdots & \ddots & \vdots & \vdots \\
0& 0 & 0 & 0 & 0 & \cdots & 0 & 1 
\end{pmatrix}
 \begin{pmatrix}
f_{0}^{n+1}\\
f_{1}^{n+1}\\
\vdots \\
f_{x_{max}-1}^{n+1}\\
f_{x_{max}}^{n+1}\\
\end{pmatrix}
\\ = 
 \begin{pmatrix}
e^{r dt} & 0 & 0 & 0 & 0 & \cdots & 0 & 0\\
-B_{+}^{\theta} & A_{2}^{\theta} & -B_{-}^{\theta} & 0 & 0 & \cdots & 0 & 0\\
0 & -B_{+}^{\theta} & A_{2}^{\theta} & -B_{-}^{\theta} & 0 & \cdots & 0 & 0 \\
\vdots& \vdots & \vdots & \vdots & \vdots & \ddots & \vdots & \vdots \\
0& 0 & 0 & 0 & 0 & \cdots & 0 & e^{r dt} 
\end{pmatrix}
 \begin{pmatrix}
f_{0}^{n}\\
f_{1}^{n}\\
\vdots \\
f_{x_{max}-1}^{n}\\
f_{x_{max}}^{n}\\
\end{pmatrix}
\end{split}
\end{align}


%%%%%%%%%%%%%%%%%%%%%%%%%%%%%%%%%%%%%%%%%%%%%%%%%%%%%%%
%%%%%%%%%%%%%%%%%%%%%%%%%%%%%%%%%%%%%%%%%%%%%%%%%%%%%%%
\newpage
\section{Neumann} \label{Npart}



$
\forall  n \in  \ldbrack  1 , N  \rdbrack , \forall  i \in  \ldbrack  1,x_{max}  \rdbrack,    \\
$

\begin{align}
\frac{df}{dx} (t_{n}, x_{0}) &= k_{1} \\
\frac{df}{dx} (t_{n},x_{0}) = \frac{f_{1}^{n}-f_{0}^{n}}{dx} &\Leftrightarrow f_{0}^{n} = f_{1}^{n} - k_{1} dx \label{formula2} 
\end{align}
\begin{align}
\frac{df}{dx} (t_{n}, x_{max}) &= k_{2} \\
\frac{df}{dx} (t_{n},x_{0}) = \frac{f_{x_{max}}^{n}-f_{x_{max}-1}^{n}}{dx} &\Leftrightarrow f_{x_{max}}^{n} = f_{x_{max}-1}^{n} + k_{2} dx \label{formula3} 
\end{align}

Using the same formula :
\begin{align} \label{formula1}
 f_{i}^{n+1} A_{1}^{(1-\theta)} +f_{i+1}^{n+1} B_{-}^{(1-\theta)}  +f_{i-1}^{n+1} B_{+}^{(1-\theta)} &=
 f_{i}^{n} A_{2}^{\theta} - f_{i+1}^{n} B_{-}^{\theta}  -f_{i-1}^{n} B_{+}^{\theta} 
 \end{align}
 
 Taking the left part of the equation \ref{formula1}, and using equation \ref{formula2} we have :
 
 \begin{align} 
 f_{i}^{n+1} A_{1}^{(1-\theta)} +f_{i+1}^{n+1} B_{-}^{(1-\theta)}  +f_{i-1}^{n+1} B_{+}^{(1-\theta)} &=
 f_{1}^{n} A_{2}^{\theta} - f_{2}^{n} B_{-}^{\theta} -  (f_{1}^{n} - k_{1} dx) B_{+}^{\theta} \\
&=  f_{1}^{n} (A_{2}^{\theta} - B_{+}^{\theta} ) - f_{2}^{n} B_{-}^{\theta} + k_{1} B_{+}^{\theta} dx
\end{align}
 
Taking the right part of the equation \ref{formula1}, and using equation \ref{formula3} we have :

 \begin{align} 
 f_{i}^{n} A_{2}^{\theta} - f_{i+1}^{n} B_{-}^{\theta}  -f_{i-1}^{n} B_{+}^{\theta}  &=
f_{x_{max}-1}^{n} A_{2}^{\theta} - f_{x_{max}}^{n} B_{-}^{\theta} -f_{x_{max}-2}^{n} B_{+}^{\theta} \\
&=  f_{x_{max}-1}^{n} (A_{2}^{\theta} - B_{-}^{\theta} ) - k_{2} B_{-}^{\theta} dx -  f_{x_{max}-2}^{n} B_{+}^{\theta} 
 \end{align}
 
  The previous formulas give us the following matrix system :
  
\begin{align}
\begin{split}
 \begin{pmatrix}
-\frac{1}{dx} & \frac{1}{dx} & 0 & 0 & 0 & \cdots & 0 & 0 & 0 & 0\\
B_{+}^{(1-\theta)} & A_{1}^{(1-\theta)} & B_{-}^{(1-\theta)} & 0 & 0 & \cdots & 0 & 0 & 0 & 0\\
0 & B_{+}^{(1-\theta)} & A_{1}^{(1-\theta)} & B_{-}^{(1-\theta)} & 0 & \cdots & 0 & 0 & 0 & 0 \\
\vdots& \vdots & \vdots & \vdots & \vdots & \ddots & \vdots & \vdots & \vdots & \vdots & \vdots \\
0 & 0 & 0 & 0 & 0 & \cdots & 0 & B_{+}^{(1-\theta)} & A_{1}^{(1-\theta)} & B_{-}^{(1-\theta)}  \\
0& 0 & 0 & 0 & 0 & \cdots & 0 & 0 & -\frac{1}{dx} & \frac{1}{dx}
\end{pmatrix}
 \begin{pmatrix}
f_{0}^{n+1}\\
f_{1}^{n+1}\\
\vdots \\
f_{x_{max}-1}^{n+1}\\
f_{x_{max}}^{n+1}\\
\end{pmatrix}
\\ =
 \begin{pmatrix}
1 & 0 & 0 & 0 & 0 & \cdots & 0 & 0 & 0 & 0 & 0 \\
B_{+}^{\theta} dx & A_{2}^{\theta}-B_{+}^{\theta}  & -B_{-}^{\theta} & 0 & 0 & \cdots & 0 & 0 & 0 & 0 & 0\\
0 & -B_{+}^{\theta} & A_{2}^{\theta} & -B_{-}^{\theta} & 0 & \cdots & 0 & 0 & 0 & 0 & 0 \\
\vdots& \vdots & \vdots & \vdots  & \vdots  & \ddots & \vdots & \vdots & \vdots & \vdots  & \vdots  \\
0& 0 & 0 & 0 & 0 & \cdots & 0 & -B_{+}^{\theta} & A_{2}^{\theta} & -B_{-}^{\theta} & 0 \\
0 & 0 & 0 & 0 & 0 & \ldots & 0 & 0 & -B_{+}^{\theta} & A_{2}^{\theta}-B_{-}^{\theta} & -B_{-}^{\theta}dx \\
0& 0 & 0 & 0 & 0 & \cdots & 0 & 0 & 0 & 0 & 1
\end{pmatrix}
 \begin{pmatrix}
k_{1}\\
f_{1}^{n}\\
\vdots \\
f_{x_{max}-1}^{n}\\
k_{2}\\
\end{pmatrix}
\end{split}
\end{align}











\end{document}
